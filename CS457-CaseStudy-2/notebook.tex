
% Default to the notebook output style

    


% Inherit from the specified cell style.




    
\documentclass[11pt]{article}

    
    
    \usepackage[T1]{fontenc}
    % Nicer default font (+ math font) than Computer Modern for most use cases
    \usepackage{mathpazo}

    % Basic figure setup, for now with no caption control since it's done
    % automatically by Pandoc (which extracts ![](path) syntax from Markdown).
    \usepackage{graphicx}
    % We will generate all images so they have a width \maxwidth. This means
    % that they will get their normal width if they fit onto the page, but
    % are scaled down if they would overflow the margins.
    \makeatletter
    \def\maxwidth{\ifdim\Gin@nat@width>\linewidth\linewidth
    \else\Gin@nat@width\fi}
    \makeatother
    \let\Oldincludegraphics\includegraphics
    % Set max figure width to be 80% of text width, for now hardcoded.
    \renewcommand{\includegraphics}[1]{\Oldincludegraphics[width=.8\maxwidth]{#1}}
    % Ensure that by default, figures have no caption (until we provide a
    % proper Figure object with a Caption API and a way to capture that
    % in the conversion process - todo).
    \usepackage{caption}
    \DeclareCaptionLabelFormat{nolabel}{}
    \captionsetup{labelformat=nolabel}

    \usepackage{adjustbox} % Used to constrain images to a maximum size 
    \usepackage{xcolor} % Allow colors to be defined
    \usepackage{enumerate} % Needed for markdown enumerations to work
    \usepackage{geometry} % Used to adjust the document margins
    \usepackage{amsmath} % Equations
    \usepackage{amssymb} % Equations
    \usepackage{textcomp} % defines textquotesingle
    % Hack from http://tex.stackexchange.com/a/47451/13684:
    \AtBeginDocument{%
        \def\PYZsq{\textquotesingle}% Upright quotes in Pygmentized code
    }
    \usepackage{upquote} % Upright quotes for verbatim code
    \usepackage{eurosym} % defines \euro
    \usepackage[mathletters]{ucs} % Extended unicode (utf-8) support
    \usepackage[utf8x]{inputenc} % Allow utf-8 characters in the tex document
    \usepackage{fancyvrb} % verbatim replacement that allows latex
    \usepackage{grffile} % extends the file name processing of package graphics 
                         % to support a larger range 
    % The hyperref package gives us a pdf with properly built
    % internal navigation ('pdf bookmarks' for the table of contents,
    % internal cross-reference links, web links for URLs, etc.)
    \usepackage{hyperref}
    \usepackage{longtable} % longtable support required by pandoc >1.10
    \usepackage{booktabs}  % table support for pandoc > 1.12.2
    \usepackage[inline]{enumitem} % IRkernel/repr support (it uses the enumerate* environment)
    \usepackage[normalem]{ulem} % ulem is needed to support strikethroughs (\sout)
                                % normalem makes italics be italics, not underlines
    \usepackage{mathrsfs}
    

    
    
    % Colors for the hyperref package
    \definecolor{urlcolor}{rgb}{0,.145,.698}
    \definecolor{linkcolor}{rgb}{.71,0.21,0.01}
    \definecolor{citecolor}{rgb}{.12,.54,.11}

    % ANSI colors
    \definecolor{ansi-black}{HTML}{3E424D}
    \definecolor{ansi-black-intense}{HTML}{282C36}
    \definecolor{ansi-red}{HTML}{E75C58}
    \definecolor{ansi-red-intense}{HTML}{B22B31}
    \definecolor{ansi-green}{HTML}{00A250}
    \definecolor{ansi-green-intense}{HTML}{007427}
    \definecolor{ansi-yellow}{HTML}{DDB62B}
    \definecolor{ansi-yellow-intense}{HTML}{B27D12}
    \definecolor{ansi-blue}{HTML}{208FFB}
    \definecolor{ansi-blue-intense}{HTML}{0065CA}
    \definecolor{ansi-magenta}{HTML}{D160C4}
    \definecolor{ansi-magenta-intense}{HTML}{A03196}
    \definecolor{ansi-cyan}{HTML}{60C6C8}
    \definecolor{ansi-cyan-intense}{HTML}{258F8F}
    \definecolor{ansi-white}{HTML}{C5C1B4}
    \definecolor{ansi-white-intense}{HTML}{A1A6B2}
    \definecolor{ansi-default-inverse-fg}{HTML}{FFFFFF}
    \definecolor{ansi-default-inverse-bg}{HTML}{000000}

    % commands and environments needed by pandoc snippets
    % extracted from the output of `pandoc -s`
    \providecommand{\tightlist}{%
      \setlength{\itemsep}{0pt}\setlength{\parskip}{0pt}}
    \DefineVerbatimEnvironment{Highlighting}{Verbatim}{commandchars=\\\{\}}
    % Add ',fontsize=\small' for more characters per line
    \newenvironment{Shaded}{}{}
    \newcommand{\KeywordTok}[1]{\textcolor[rgb]{0.00,0.44,0.13}{\textbf{{#1}}}}
    \newcommand{\DataTypeTok}[1]{\textcolor[rgb]{0.56,0.13,0.00}{{#1}}}
    \newcommand{\DecValTok}[1]{\textcolor[rgb]{0.25,0.63,0.44}{{#1}}}
    \newcommand{\BaseNTok}[1]{\textcolor[rgb]{0.25,0.63,0.44}{{#1}}}
    \newcommand{\FloatTok}[1]{\textcolor[rgb]{0.25,0.63,0.44}{{#1}}}
    \newcommand{\CharTok}[1]{\textcolor[rgb]{0.25,0.44,0.63}{{#1}}}
    \newcommand{\StringTok}[1]{\textcolor[rgb]{0.25,0.44,0.63}{{#1}}}
    \newcommand{\CommentTok}[1]{\textcolor[rgb]{0.38,0.63,0.69}{\textit{{#1}}}}
    \newcommand{\OtherTok}[1]{\textcolor[rgb]{0.00,0.44,0.13}{{#1}}}
    \newcommand{\AlertTok}[1]{\textcolor[rgb]{1.00,0.00,0.00}{\textbf{{#1}}}}
    \newcommand{\FunctionTok}[1]{\textcolor[rgb]{0.02,0.16,0.49}{{#1}}}
    \newcommand{\RegionMarkerTok}[1]{{#1}}
    \newcommand{\ErrorTok}[1]{\textcolor[rgb]{1.00,0.00,0.00}{\textbf{{#1}}}}
    \newcommand{\NormalTok}[1]{{#1}}
    
    % Additional commands for more recent versions of Pandoc
    \newcommand{\ConstantTok}[1]{\textcolor[rgb]{0.53,0.00,0.00}{{#1}}}
    \newcommand{\SpecialCharTok}[1]{\textcolor[rgb]{0.25,0.44,0.63}{{#1}}}
    \newcommand{\VerbatimStringTok}[1]{\textcolor[rgb]{0.25,0.44,0.63}{{#1}}}
    \newcommand{\SpecialStringTok}[1]{\textcolor[rgb]{0.73,0.40,0.53}{{#1}}}
    \newcommand{\ImportTok}[1]{{#1}}
    \newcommand{\DocumentationTok}[1]{\textcolor[rgb]{0.73,0.13,0.13}{\textit{{#1}}}}
    \newcommand{\AnnotationTok}[1]{\textcolor[rgb]{0.38,0.63,0.69}{\textbf{\textit{{#1}}}}}
    \newcommand{\CommentVarTok}[1]{\textcolor[rgb]{0.38,0.63,0.69}{\textbf{\textit{{#1}}}}}
    \newcommand{\VariableTok}[1]{\textcolor[rgb]{0.10,0.09,0.49}{{#1}}}
    \newcommand{\ControlFlowTok}[1]{\textcolor[rgb]{0.00,0.44,0.13}{\textbf{{#1}}}}
    \newcommand{\OperatorTok}[1]{\textcolor[rgb]{0.40,0.40,0.40}{{#1}}}
    \newcommand{\BuiltInTok}[1]{{#1}}
    \newcommand{\ExtensionTok}[1]{{#1}}
    \newcommand{\PreprocessorTok}[1]{\textcolor[rgb]{0.74,0.48,0.00}{{#1}}}
    \newcommand{\AttributeTok}[1]{\textcolor[rgb]{0.49,0.56,0.16}{{#1}}}
    \newcommand{\InformationTok}[1]{\textcolor[rgb]{0.38,0.63,0.69}{\textbf{\textit{{#1}}}}}
    \newcommand{\WarningTok}[1]{\textcolor[rgb]{0.38,0.63,0.69}{\textbf{\textit{{#1}}}}}
    
    
    % Define a nice break command that doesn't care if a line doesn't already
    % exist.
    \def\br{\hspace*{\fill} \\* }
    % Math Jax compatibility definitions
    \def\gt{>}
    \def\lt{<}
    \let\Oldtex\TeX
    \let\Oldlatex\LaTeX
    \renewcommand{\TeX}{\textrm{\Oldtex}}
    \renewcommand{\LaTeX}{\textrm{\Oldlatex}}
    % Document parameters
    % Document title
    \title{notebook}
    
    
    
    
    

    % Pygments definitions
    
\makeatletter
\def\PY@reset{\let\PY@it=\relax \let\PY@bf=\relax%
    \let\PY@ul=\relax \let\PY@tc=\relax%
    \let\PY@bc=\relax \let\PY@ff=\relax}
\def\PY@tok#1{\csname PY@tok@#1\endcsname}
\def\PY@toks#1+{\ifx\relax#1\empty\else%
    \PY@tok{#1}\expandafter\PY@toks\fi}
\def\PY@do#1{\PY@bc{\PY@tc{\PY@ul{%
    \PY@it{\PY@bf{\PY@ff{#1}}}}}}}
\def\PY#1#2{\PY@reset\PY@toks#1+\relax+\PY@do{#2}}

\expandafter\def\csname PY@tok@w\endcsname{\def\PY@tc##1{\textcolor[rgb]{0.73,0.73,0.73}{##1}}}
\expandafter\def\csname PY@tok@c\endcsname{\let\PY@it=\textit\def\PY@tc##1{\textcolor[rgb]{0.25,0.50,0.50}{##1}}}
\expandafter\def\csname PY@tok@cp\endcsname{\def\PY@tc##1{\textcolor[rgb]{0.74,0.48,0.00}{##1}}}
\expandafter\def\csname PY@tok@k\endcsname{\let\PY@bf=\textbf\def\PY@tc##1{\textcolor[rgb]{0.00,0.50,0.00}{##1}}}
\expandafter\def\csname PY@tok@kp\endcsname{\def\PY@tc##1{\textcolor[rgb]{0.00,0.50,0.00}{##1}}}
\expandafter\def\csname PY@tok@kt\endcsname{\def\PY@tc##1{\textcolor[rgb]{0.69,0.00,0.25}{##1}}}
\expandafter\def\csname PY@tok@o\endcsname{\def\PY@tc##1{\textcolor[rgb]{0.40,0.40,0.40}{##1}}}
\expandafter\def\csname PY@tok@ow\endcsname{\let\PY@bf=\textbf\def\PY@tc##1{\textcolor[rgb]{0.67,0.13,1.00}{##1}}}
\expandafter\def\csname PY@tok@nb\endcsname{\def\PY@tc##1{\textcolor[rgb]{0.00,0.50,0.00}{##1}}}
\expandafter\def\csname PY@tok@nf\endcsname{\def\PY@tc##1{\textcolor[rgb]{0.00,0.00,1.00}{##1}}}
\expandafter\def\csname PY@tok@nc\endcsname{\let\PY@bf=\textbf\def\PY@tc##1{\textcolor[rgb]{0.00,0.00,1.00}{##1}}}
\expandafter\def\csname PY@tok@nn\endcsname{\let\PY@bf=\textbf\def\PY@tc##1{\textcolor[rgb]{0.00,0.00,1.00}{##1}}}
\expandafter\def\csname PY@tok@ne\endcsname{\let\PY@bf=\textbf\def\PY@tc##1{\textcolor[rgb]{0.82,0.25,0.23}{##1}}}
\expandafter\def\csname PY@tok@nv\endcsname{\def\PY@tc##1{\textcolor[rgb]{0.10,0.09,0.49}{##1}}}
\expandafter\def\csname PY@tok@no\endcsname{\def\PY@tc##1{\textcolor[rgb]{0.53,0.00,0.00}{##1}}}
\expandafter\def\csname PY@tok@nl\endcsname{\def\PY@tc##1{\textcolor[rgb]{0.63,0.63,0.00}{##1}}}
\expandafter\def\csname PY@tok@ni\endcsname{\let\PY@bf=\textbf\def\PY@tc##1{\textcolor[rgb]{0.60,0.60,0.60}{##1}}}
\expandafter\def\csname PY@tok@na\endcsname{\def\PY@tc##1{\textcolor[rgb]{0.49,0.56,0.16}{##1}}}
\expandafter\def\csname PY@tok@nt\endcsname{\let\PY@bf=\textbf\def\PY@tc##1{\textcolor[rgb]{0.00,0.50,0.00}{##1}}}
\expandafter\def\csname PY@tok@nd\endcsname{\def\PY@tc##1{\textcolor[rgb]{0.67,0.13,1.00}{##1}}}
\expandafter\def\csname PY@tok@s\endcsname{\def\PY@tc##1{\textcolor[rgb]{0.73,0.13,0.13}{##1}}}
\expandafter\def\csname PY@tok@sd\endcsname{\let\PY@it=\textit\def\PY@tc##1{\textcolor[rgb]{0.73,0.13,0.13}{##1}}}
\expandafter\def\csname PY@tok@si\endcsname{\let\PY@bf=\textbf\def\PY@tc##1{\textcolor[rgb]{0.73,0.40,0.53}{##1}}}
\expandafter\def\csname PY@tok@se\endcsname{\let\PY@bf=\textbf\def\PY@tc##1{\textcolor[rgb]{0.73,0.40,0.13}{##1}}}
\expandafter\def\csname PY@tok@sr\endcsname{\def\PY@tc##1{\textcolor[rgb]{0.73,0.40,0.53}{##1}}}
\expandafter\def\csname PY@tok@ss\endcsname{\def\PY@tc##1{\textcolor[rgb]{0.10,0.09,0.49}{##1}}}
\expandafter\def\csname PY@tok@sx\endcsname{\def\PY@tc##1{\textcolor[rgb]{0.00,0.50,0.00}{##1}}}
\expandafter\def\csname PY@tok@m\endcsname{\def\PY@tc##1{\textcolor[rgb]{0.40,0.40,0.40}{##1}}}
\expandafter\def\csname PY@tok@gh\endcsname{\let\PY@bf=\textbf\def\PY@tc##1{\textcolor[rgb]{0.00,0.00,0.50}{##1}}}
\expandafter\def\csname PY@tok@gu\endcsname{\let\PY@bf=\textbf\def\PY@tc##1{\textcolor[rgb]{0.50,0.00,0.50}{##1}}}
\expandafter\def\csname PY@tok@gd\endcsname{\def\PY@tc##1{\textcolor[rgb]{0.63,0.00,0.00}{##1}}}
\expandafter\def\csname PY@tok@gi\endcsname{\def\PY@tc##1{\textcolor[rgb]{0.00,0.63,0.00}{##1}}}
\expandafter\def\csname PY@tok@gr\endcsname{\def\PY@tc##1{\textcolor[rgb]{1.00,0.00,0.00}{##1}}}
\expandafter\def\csname PY@tok@ge\endcsname{\let\PY@it=\textit}
\expandafter\def\csname PY@tok@gs\endcsname{\let\PY@bf=\textbf}
\expandafter\def\csname PY@tok@gp\endcsname{\let\PY@bf=\textbf\def\PY@tc##1{\textcolor[rgb]{0.00,0.00,0.50}{##1}}}
\expandafter\def\csname PY@tok@go\endcsname{\def\PY@tc##1{\textcolor[rgb]{0.53,0.53,0.53}{##1}}}
\expandafter\def\csname PY@tok@gt\endcsname{\def\PY@tc##1{\textcolor[rgb]{0.00,0.27,0.87}{##1}}}
\expandafter\def\csname PY@tok@err\endcsname{\def\PY@bc##1{\setlength{\fboxsep}{0pt}\fcolorbox[rgb]{1.00,0.00,0.00}{1,1,1}{\strut ##1}}}
\expandafter\def\csname PY@tok@kc\endcsname{\let\PY@bf=\textbf\def\PY@tc##1{\textcolor[rgb]{0.00,0.50,0.00}{##1}}}
\expandafter\def\csname PY@tok@kd\endcsname{\let\PY@bf=\textbf\def\PY@tc##1{\textcolor[rgb]{0.00,0.50,0.00}{##1}}}
\expandafter\def\csname PY@tok@kn\endcsname{\let\PY@bf=\textbf\def\PY@tc##1{\textcolor[rgb]{0.00,0.50,0.00}{##1}}}
\expandafter\def\csname PY@tok@kr\endcsname{\let\PY@bf=\textbf\def\PY@tc##1{\textcolor[rgb]{0.00,0.50,0.00}{##1}}}
\expandafter\def\csname PY@tok@bp\endcsname{\def\PY@tc##1{\textcolor[rgb]{0.00,0.50,0.00}{##1}}}
\expandafter\def\csname PY@tok@fm\endcsname{\def\PY@tc##1{\textcolor[rgb]{0.00,0.00,1.00}{##1}}}
\expandafter\def\csname PY@tok@vc\endcsname{\def\PY@tc##1{\textcolor[rgb]{0.10,0.09,0.49}{##1}}}
\expandafter\def\csname PY@tok@vg\endcsname{\def\PY@tc##1{\textcolor[rgb]{0.10,0.09,0.49}{##1}}}
\expandafter\def\csname PY@tok@vi\endcsname{\def\PY@tc##1{\textcolor[rgb]{0.10,0.09,0.49}{##1}}}
\expandafter\def\csname PY@tok@vm\endcsname{\def\PY@tc##1{\textcolor[rgb]{0.10,0.09,0.49}{##1}}}
\expandafter\def\csname PY@tok@sa\endcsname{\def\PY@tc##1{\textcolor[rgb]{0.73,0.13,0.13}{##1}}}
\expandafter\def\csname PY@tok@sb\endcsname{\def\PY@tc##1{\textcolor[rgb]{0.73,0.13,0.13}{##1}}}
\expandafter\def\csname PY@tok@sc\endcsname{\def\PY@tc##1{\textcolor[rgb]{0.73,0.13,0.13}{##1}}}
\expandafter\def\csname PY@tok@dl\endcsname{\def\PY@tc##1{\textcolor[rgb]{0.73,0.13,0.13}{##1}}}
\expandafter\def\csname PY@tok@s2\endcsname{\def\PY@tc##1{\textcolor[rgb]{0.73,0.13,0.13}{##1}}}
\expandafter\def\csname PY@tok@sh\endcsname{\def\PY@tc##1{\textcolor[rgb]{0.73,0.13,0.13}{##1}}}
\expandafter\def\csname PY@tok@s1\endcsname{\def\PY@tc##1{\textcolor[rgb]{0.73,0.13,0.13}{##1}}}
\expandafter\def\csname PY@tok@mb\endcsname{\def\PY@tc##1{\textcolor[rgb]{0.40,0.40,0.40}{##1}}}
\expandafter\def\csname PY@tok@mf\endcsname{\def\PY@tc##1{\textcolor[rgb]{0.40,0.40,0.40}{##1}}}
\expandafter\def\csname PY@tok@mh\endcsname{\def\PY@tc##1{\textcolor[rgb]{0.40,0.40,0.40}{##1}}}
\expandafter\def\csname PY@tok@mi\endcsname{\def\PY@tc##1{\textcolor[rgb]{0.40,0.40,0.40}{##1}}}
\expandafter\def\csname PY@tok@il\endcsname{\def\PY@tc##1{\textcolor[rgb]{0.40,0.40,0.40}{##1}}}
\expandafter\def\csname PY@tok@mo\endcsname{\def\PY@tc##1{\textcolor[rgb]{0.40,0.40,0.40}{##1}}}
\expandafter\def\csname PY@tok@ch\endcsname{\let\PY@it=\textit\def\PY@tc##1{\textcolor[rgb]{0.25,0.50,0.50}{##1}}}
\expandafter\def\csname PY@tok@cm\endcsname{\let\PY@it=\textit\def\PY@tc##1{\textcolor[rgb]{0.25,0.50,0.50}{##1}}}
\expandafter\def\csname PY@tok@cpf\endcsname{\let\PY@it=\textit\def\PY@tc##1{\textcolor[rgb]{0.25,0.50,0.50}{##1}}}
\expandafter\def\csname PY@tok@c1\endcsname{\let\PY@it=\textit\def\PY@tc##1{\textcolor[rgb]{0.25,0.50,0.50}{##1}}}
\expandafter\def\csname PY@tok@cs\endcsname{\let\PY@it=\textit\def\PY@tc##1{\textcolor[rgb]{0.25,0.50,0.50}{##1}}}

\def\PYZbs{\char`\\}
\def\PYZus{\char`\_}
\def\PYZob{\char`\{}
\def\PYZcb{\char`\}}
\def\PYZca{\char`\^}
\def\PYZam{\char`\&}
\def\PYZlt{\char`\<}
\def\PYZgt{\char`\>}
\def\PYZsh{\char`\#}
\def\PYZpc{\char`\%}
\def\PYZdl{\char`\$}
\def\PYZhy{\char`\-}
\def\PYZsq{\char`\'}
\def\PYZdq{\char`\"}
\def\PYZti{\char`\~}
% for compatibility with earlier versions
\def\PYZat{@}
\def\PYZlb{[}
\def\PYZrb{]}
\makeatother


    % Exact colors from NB
    \definecolor{incolor}{rgb}{0.0, 0.0, 0.5}
    \definecolor{outcolor}{rgb}{0.545, 0.0, 0.0}



    
    % Prevent overflowing lines due to hard-to-break entities
    \sloppy 
    % Setup hyperref package
    \hypersetup{
      breaklinks=true,  % so long urls are correctly broken across lines
      colorlinks=true,
      urlcolor=urlcolor,
      linkcolor=linkcolor,
      citecolor=citecolor,
      }
    % Slightly bigger margins than the latex defaults
    
    \geometry{verbose,tmargin=1in,bmargin=1in,lmargin=1in,rmargin=1in}
    
    

    \begin{document}
    
    
    \maketitle
    
    

    
    \hypertarget{your-full-name-m-shakaib-saleem}{%
\subsubsection{Your Full Name: M Shakaib
Saleem}\label{your-full-name-m-shakaib-saleem}}

\hypertarget{your-id-ms01036}{%
\subsubsection{Your ID: ms01036}\label{your-id-ms01036}}

\hypertarget{your-email-address-ms01036st.habib.edu.pk}{%
\subsubsection{Your Email Address:
ms01036@st.habib.edu.pk}\label{your-email-address-ms01036st.habib.edu.pk}}

    \hypertarget{credit-card-applications}{%
\subsection{1. Credit card
applications}\label{credit-card-applications}}

Commercial banks receive a lot of applications for credit cards. Many of
them get rejected for many reasons, like high loan balances, low income
levels, or too many inquiries on an individual's credit report, for
example. Manually analyzing these applications is mundane, error-prone,
and time-consuming (and time is money!). Luckily, this task can be
automated with the power of machine learning and pretty much every
commercial bank does so nowadays. In this notebook, we will build an
automatic credit card approval predictor using machine learning
techniques, just like the real banks do!

We'll use the Credit Card Approval dataset from the UCI Machine Learning
Repository. The structure of this notebook is as follows:

First, we will start off by loading and viewing the dataset.

We will see that the dataset has a mixture of both numerical and
non-numerical features, that it contains values from different ranges,
plus that it contains a number of missing entries.

We will have to preprocess the dataset to ensure the machine learning
model we choose can make good predictions.

After our data is in good shape, we will do some exploratory data
analysis to build our intuitions.

Finally, we will build a machine learning model that can predict if an
individual's application for a credit card will be accepted.

First, loading and viewing the dataset. We find that since this data is
confidential, the contributor of the dataset has anonymized the feature
names.

    \begin{Verbatim}[commandchars=\\\{\}]
{\color{incolor}In [{\color{incolor}1}]:} \PY{c+c1}{\PYZsh{} Import pandas}
        \PY{c+c1}{\PYZsh{} ... YOUR CODE FOR TASK 1 ...}
        \PY{k+kn}{import} \PY{n+nn}{pandas} \PY{k}{as} \PY{n+nn}{pd}
        
        \PY{c+c1}{\PYZsh{} Load dataset}
        \PY{n}{cc\PYZus{}apps} \PY{o}{=} \PY{n}{pd}\PY{o}{.}\PY{n}{read\PYZus{}csv}\PY{p}{(}\PY{l+s+s2}{\PYZdq{}}\PY{l+s+s2}{datasets/cc\PYZus{}approvals.data}\PY{l+s+s2}{\PYZdq{}}\PY{p}{,}\PY{n}{header}\PY{o}{=}\PY{k+kc}{None}\PY{p}{)}
        
        \PY{c+c1}{\PYZsh{} Inspect data}
        \PY{c+c1}{\PYZsh{} ... YOUR CODE FOR TASK 1 ...}
        \PY{n}{display}\PY{p}{(}\PY{n}{cc\PYZus{}apps}\PY{o}{.}\PY{n}{head}\PY{p}{(}\PY{p}{)}\PY{p}{)}
\end{Verbatim}

    
    \begin{verbatim}
  0      1      2  3  4  5  6     7  8  9   10 11 12     13   14 15
0  b  30.83  0.000  u  g  w  v  1.25  t  t   1  f  g  00202    0  +
1  a  58.67  4.460  u  g  q  h  3.04  t  t   6  f  g  00043  560  +
2  a  24.50  0.500  u  g  q  h  1.50  t  f   0  f  g  00280  824  +
3  b  27.83  1.540  u  g  w  v  3.75  t  t   5  t  g  00100    3  +
4  b  20.17  5.625  u  g  w  v  1.71  t  f   0  f  s  00120    0  +
    \end{verbatim}

    
    \hypertarget{findings}{%
\subparagraph{Findings:}\label{findings}}

We can see that each record in our dataset has 16 attributes and that
these attributes are of various types: numerical, categorical and string
object. The last one is the class attribute which seems to have been
encoded as + for `Yes' outcome.

    \hypertarget{inspecting-the-applications}{%
\subsection{2. Inspecting the
applications}\label{inspecting-the-applications}}

The output may appear a bit confusing at its first sight, but let's try
to figure out the most important features of a credit card application.
The features of this dataset have been anonymized to protect the
privacy, but this blog gives us a pretty good overview of the probable
features. The probable features in a typical credit card application are
Gender, Age, Debt, Married, BankCustomer, EducationLevel, Ethnicity,
YearsEmployed, PriorDefault, Employed, CreditScore, DriversLicense,
Citizen, ZipCode, Income and finally the ApprovalStatus. This gives us a
pretty good starting point, and we can map these features with respect
to the columns in the output.

As we can see from our first glance at the data, the dataset has a
mixture of numerical and non-numerical features. This can be fixed with
some preprocessing, but before we do that, let's learn about the dataset
a bit more to see if there are other dataset issues that need to be
fixed.

    \begin{Verbatim}[commandchars=\\\{\}]
{\color{incolor}In [{\color{incolor}2}]:} \PY{c+c1}{\PYZsh{} Print summary statistics}
        \PY{n}{cc\PYZus{}apps\PYZus{}description} \PY{o}{=} \PY{n}{cc\PYZus{}apps}\PY{o}{.}\PY{n}{describe}\PY{p}{(}\PY{p}{)}
        \PY{n}{display}\PY{p}{(}\PY{n}{cc\PYZus{}apps\PYZus{}description}\PY{p}{)}
        
        \PY{c+c1}{\PYZsh{} Print DataFrame information}
        \PY{n}{cc\PYZus{}apps\PYZus{}info} \PY{o}{=} \PY{n}{cc\PYZus{}apps}\PY{o}{.}\PY{n}{info}\PY{p}{(}\PY{p}{)}
        \PY{n}{display}\PY{p}{(}\PY{n}{cc\PYZus{}apps\PYZus{}info}\PY{p}{)}
        
        \PY{c+c1}{\PYZsh{} Inspect missing values in the dataset}
        \PY{c+c1}{\PYZsh{} ... YOUR CODE FOR TASK 2 ...}
        \PY{n}{display}\PY{p}{(}\PY{n}{cc\PYZus{}apps}\PY{o}{.}\PY{n}{head}\PY{p}{(}\PY{l+m+mi}{30}\PY{p}{)}\PY{p}{)}
        \PY{n}{display}\PY{p}{(}\PY{n}{cc\PYZus{}apps}\PY{o}{.}\PY{n}{tail}\PY{p}{(}\PY{l+m+mi}{30}\PY{p}{)}\PY{p}{)}
\end{Verbatim}

    
    \begin{verbatim}
               2           7          10             14
count  690.000000  690.000000  690.00000     690.000000
mean     4.758725    2.223406    2.40000    1017.385507
std      4.978163    3.346513    4.86294    5210.102598
min      0.000000    0.000000    0.00000       0.000000
25%      1.000000    0.165000    0.00000       0.000000
50%      2.750000    1.000000    0.00000       5.000000
75%      7.207500    2.625000    3.00000     395.500000
max     28.000000   28.500000   67.00000  100000.000000
    \end{verbatim}

    
    \begin{Verbatim}[commandchars=\\\{\}]
<class 'pandas.core.frame.DataFrame'>
RangeIndex: 690 entries, 0 to 689
Data columns (total 16 columns):
0     690 non-null object
1     690 non-null object
2     690 non-null float64
3     690 non-null object
4     690 non-null object
5     690 non-null object
6     690 non-null object
7     690 non-null float64
8     690 non-null object
9     690 non-null object
10    690 non-null int64
11    690 non-null object
12    690 non-null object
13    690 non-null object
14    690 non-null int64
15    690 non-null object
dtypes: float64(2), int64(2), object(12)
memory usage: 86.3+ KB

    \end{Verbatim}

    
    \begin{verbatim}
None
    \end{verbatim}

    
    
    \begin{verbatim}
   0      1       2  3  4   5   6       7  8  9   10 11 12     13     14 15
0   b  30.83   0.000  u  g   w   v   1.250  t  t   1  f  g  00202      0  +
1   a  58.67   4.460  u  g   q   h   3.040  t  t   6  f  g  00043    560  +
2   a  24.50   0.500  u  g   q   h   1.500  t  f   0  f  g  00280    824  +
3   b  27.83   1.540  u  g   w   v   3.750  t  t   5  t  g  00100      3  +
4   b  20.17   5.625  u  g   w   v   1.710  t  f   0  f  s  00120      0  +
5   b  32.08   4.000  u  g   m   v   2.500  t  f   0  t  g  00360      0  +
6   b  33.17   1.040  u  g   r   h   6.500  t  f   0  t  g  00164  31285  +
7   a  22.92  11.585  u  g  cc   v   0.040  t  f   0  f  g  00080   1349  +
8   b  54.42   0.500  y  p   k   h   3.960  t  f   0  f  g  00180    314  +
9   b  42.50   4.915  y  p   w   v   3.165  t  f   0  t  g  00052   1442  +
10  b  22.08   0.830  u  g   c   h   2.165  f  f   0  t  g  00128      0  +
11  b  29.92   1.835  u  g   c   h   4.335  t  f   0  f  g  00260    200  +
12  a  38.25   6.000  u  g   k   v   1.000  t  f   0  t  g  00000      0  +
13  b  48.08   6.040  u  g   k   v   0.040  f  f   0  f  g  00000   2690  +
14  a  45.83  10.500  u  g   q   v   5.000  t  t   7  t  g  00000      0  +
15  b  36.67   4.415  y  p   k   v   0.250  t  t  10  t  g  00320      0  +
16  b  28.25   0.875  u  g   m   v   0.960  t  t   3  t  g  00396      0  +
17  a  23.25   5.875  u  g   q   v   3.170  t  t  10  f  g  00120    245  +
18  b  21.83   0.250  u  g   d   h   0.665  t  f   0  t  g  00000      0  +
19  a  19.17   8.585  u  g  cc   h   0.750  t  t   7  f  g  00096      0  +
20  b  25.00  11.250  u  g   c   v   2.500  t  t  17  f  g  00200   1208  +
21  b  23.25   1.000  u  g   c   v   0.835  t  f   0  f  s  00300      0  +
22  a  47.75   8.000  u  g   c   v   7.875  t  t   6  t  g  00000   1260  +
23  a  27.42  14.500  u  g   x   h   3.085  t  t   1  f  g  00120     11  +
24  a  41.17   6.500  u  g   q   v   0.500  t  t   3  t  g  00145      0  +
25  a  15.83   0.585  u  g   c   h   1.500  t  t   2  f  g  00100      0  +
26  a  47.00  13.000  u  g   i  bb   5.165  t  t   9  t  g  00000      0  +
27  b  56.58  18.500  u  g   d  bb  15.000  t  t  17  t  g  00000      0  +
28  b  57.42   8.500  u  g   e   h   7.000  t  t   3  f  g  00000      0  +
29  b  42.08   1.040  u  g   w   v   5.000  t  t   6  t  g  00500  10000  +
    \end{verbatim}

    
    
    \begin{verbatim}
    0      1       2  3  4   5   6      7  8  9   10 11 12     13   14 15
660  b  22.25   9.000  u  g  aa   v  0.085  f  f   0  f  g  00000    0  -
661  b  29.83   3.500  u  g   c   v  0.165  f  f   0  f  g  00216    0  -
662  a  23.50   1.500  u  g   w   v  0.875  f  f   0  t  g  00160    0  -
663  b  32.08   4.000  y  p  cc   v  1.500  f  f   0  t  g  00120    0  -
664  b  31.08   1.500  y  p   w   v  0.040  f  f   0  f  s  00160    0  -
665  b  31.83   0.040  y  p   m   v  0.040  f  f   0  f  g  00000    0  -
666  a  21.75  11.750  u  g   c   v  0.250  f  f   0  t  g  00180    0  -
667  a  17.92   0.540  u  g   c   v  1.750  f  t   1  t  g  00080    5  -
668  b  30.33   0.500  u  g   d   h  0.085  f  f   0  t  s  00252    0  -
669  b  51.83   2.040  y  p  ff  ff  1.500  f  f   0  f  g  00120    1  -
670  b  47.17   5.835  u  g   w   v  5.500  f  f   0  f  g  00465  150  -
671  b  25.83  12.835  u  g  cc   v  0.500  f  f   0  f  g  00000    2  -
672  a  50.25   0.835  u  g  aa   v  0.500  f  f   0  t  g  00240  117  -
673  ?  29.50   2.000  y  p   e   h  2.000  f  f   0  f  g  00256   17  -
674  a  37.33   2.500  u  g   i   h  0.210  f  f   0  f  g  00260  246  -
675  a  41.58   1.040  u  g  aa   v  0.665  f  f   0  f  g  00240  237  -
676  a  30.58  10.665  u  g   q   h  0.085  f  t  12  t  g  00129    3  -
677  b  19.42   7.250  u  g   m   v  0.040  f  t   1  f  g  00100    1  -
678  a  17.92  10.210  u  g  ff  ff  0.000  f  f   0  f  g  00000   50  -
679  a  20.08   1.250  u  g   c   v  0.000  f  f   0  f  g  00000    0  -
680  b  19.50   0.290  u  g   k   v  0.290  f  f   0  f  g  00280  364  -
681  b  27.83   1.000  y  p   d   h  3.000  f  f   0  f  g  00176  537  -
682  b  17.08   3.290  u  g   i   v  0.335  f  f   0  t  g  00140    2  -
683  b  36.42   0.750  y  p   d   v  0.585  f  f   0  f  g  00240    3  -
684  b  40.58   3.290  u  g   m   v  3.500  f  f   0  t  s  00400    0  -
685  b  21.08  10.085  y  p   e   h  1.250  f  f   0  f  g  00260    0  -
686  a  22.67   0.750  u  g   c   v  2.000  f  t   2  t  g  00200  394  -
687  a  25.25  13.500  y  p  ff  ff  2.000  f  t   1  t  g  00200    1  -
688  b  17.92   0.205  u  g  aa   v  0.040  f  f   0  f  g  00280  750  -
689  b  35.00   3.375  u  g   c   h  8.290  f  f   0  t  g  00000    0  -
    \end{verbatim}

    
    \hypertarget{findings}{%
\subparagraph{Findings:}\label{findings}}

From the description summary that lists only the numeric columns, we can
see that there are 690 records. These columns are 2,7,10, and 14; column
1 despite having numeric values, has not been included (which may
indicate missing values represented non-numerically) and column 13 has
numeric data stored as a string object and is hence not included.

The Info part tells us that of the 4 numeric columns, 2 are float and 2
are integer type, while the remaining 12 are string object types, some
of which may be categorical.

Viewing the first and last few records shows that (in row\# 673) some
missing data has been represented as `?' which changes the column type
to non-numeric.

    \hypertarget{handling-the-missing-values-part-i}{%
\subsection{3. Handling the missing values (part
i)}\label{handling-the-missing-values-part-i}}

We've uncovered some issues that will affect the performance of our
machine learning model(s) if they go unchanged:

Our dataset contains both numeric and non-numeric data (specifically
data that are of float64, int64 and object types). Specifically, the
features 2, 7, 10 and 14 contain numeric values (of types float64,
float64, int64 and int64 respectively) and all the other features
contain non-numeric values.

The dataset also contains values from several ranges. Some features have
a value range of 0 - 28, some have a range of 2 - 67, and some have a
range of 1017 - 100000. Apart from these, we can get useful statistical
information (like mean, max, and min) about the features that have
numerical values.

Finally, the dataset has missing values, which we'll take care of in
this task. The missing values in the dataset are labeled with `?', which
can be seen in the last cell's output.

Now, let's temporarily replace these missing value question marks with
NaN.

    \begin{Verbatim}[commandchars=\\\{\}]
{\color{incolor}In [{\color{incolor}3}]:} \PY{c+c1}{\PYZsh{} Import numpy}
        \PY{c+c1}{\PYZsh{} ... YOUR CODE FOR TASK 3 ...}
        \PY{k+kn}{import} \PY{n+nn}{numpy} \PY{k}{as} \PY{n+nn}{np}
        
        \PY{c+c1}{\PYZsh{} Inspect missing values in the dataset}
        \PY{n}{display}\PY{p}{(}\PY{n}{cc\PYZus{}apps}\PY{o}{.}\PY{n}{tail}\PY{p}{(}\PY{l+m+mi}{17}\PY{p}{)}\PY{p}{)}
        
        \PY{c+c1}{\PYZsh{} Replace the \PYZsq{}?\PYZsq{}s with NaN}
        \PY{n}{cc\PYZus{}apps} \PY{o}{=} \PY{n}{cc\PYZus{}apps}\PY{o}{.}\PY{n}{replace}\PY{p}{(}\PY{l+s+s1}{\PYZsq{}}\PY{l+s+s1}{?}\PY{l+s+s1}{\PYZsq{}}\PY{p}{,}\PY{n}{np}\PY{o}{.}\PY{n}{NaN}\PY{p}{)}
        
        \PY{c+c1}{\PYZsh{} Inspect the missing values again}
        \PY{c+c1}{\PYZsh{} ... YOUR CODE FOR TASK 3 ...\PYZhy{}}
        \PY{n}{display}\PY{p}{(}\PY{n}{cc\PYZus{}apps}\PY{o}{.}\PY{n}{tail}\PY{p}{(}\PY{l+m+mi}{17}\PY{p}{)}\PY{p}{)}
\end{Verbatim}

    
    \begin{verbatim}
    0      1       2  3  4   5   6      7  8  9   10 11 12     13   14 15
673  ?  29.50   2.000  y  p   e   h  2.000  f  f   0  f  g  00256   17  -
674  a  37.33   2.500  u  g   i   h  0.210  f  f   0  f  g  00260  246  -
675  a  41.58   1.040  u  g  aa   v  0.665  f  f   0  f  g  00240  237  -
676  a  30.58  10.665  u  g   q   h  0.085  f  t  12  t  g  00129    3  -
677  b  19.42   7.250  u  g   m   v  0.040  f  t   1  f  g  00100    1  -
678  a  17.92  10.210  u  g  ff  ff  0.000  f  f   0  f  g  00000   50  -
679  a  20.08   1.250  u  g   c   v  0.000  f  f   0  f  g  00000    0  -
680  b  19.50   0.290  u  g   k   v  0.290  f  f   0  f  g  00280  364  -
681  b  27.83   1.000  y  p   d   h  3.000  f  f   0  f  g  00176  537  -
682  b  17.08   3.290  u  g   i   v  0.335  f  f   0  t  g  00140    2  -
683  b  36.42   0.750  y  p   d   v  0.585  f  f   0  f  g  00240    3  -
684  b  40.58   3.290  u  g   m   v  3.500  f  f   0  t  s  00400    0  -
685  b  21.08  10.085  y  p   e   h  1.250  f  f   0  f  g  00260    0  -
686  a  22.67   0.750  u  g   c   v  2.000  f  t   2  t  g  00200  394  -
687  a  25.25  13.500  y  p  ff  ff  2.000  f  t   1  t  g  00200    1  -
688  b  17.92   0.205  u  g  aa   v  0.040  f  f   0  f  g  00280  750  -
689  b  35.00   3.375  u  g   c   h  8.290  f  f   0  t  g  00000    0  -
    \end{verbatim}

    
    
    \begin{verbatim}
      0      1       2  3  4   5   6      7  8  9   10 11 12     13   14 15
673  NaN  29.50   2.000  y  p   e   h  2.000  f  f   0  f  g  00256   17  -
674    a  37.33   2.500  u  g   i   h  0.210  f  f   0  f  g  00260  246  -
675    a  41.58   1.040  u  g  aa   v  0.665  f  f   0  f  g  00240  237  -
676    a  30.58  10.665  u  g   q   h  0.085  f  t  12  t  g  00129    3  -
677    b  19.42   7.250  u  g   m   v  0.040  f  t   1  f  g  00100    1  -
678    a  17.92  10.210  u  g  ff  ff  0.000  f  f   0  f  g  00000   50  -
679    a  20.08   1.250  u  g   c   v  0.000  f  f   0  f  g  00000    0  -
680    b  19.50   0.290  u  g   k   v  0.290  f  f   0  f  g  00280  364  -
681    b  27.83   1.000  y  p   d   h  3.000  f  f   0  f  g  00176  537  -
682    b  17.08   3.290  u  g   i   v  0.335  f  f   0  t  g  00140    2  -
683    b  36.42   0.750  y  p   d   v  0.585  f  f   0  f  g  00240    3  -
684    b  40.58   3.290  u  g   m   v  3.500  f  f   0  t  s  00400    0  -
685    b  21.08  10.085  y  p   e   h  1.250  f  f   0  f  g  00260    0  -
686    a  22.67   0.750  u  g   c   v  2.000  f  t   2  t  g  00200  394  -
687    a  25.25  13.500  y  p  ff  ff  2.000  f  t   1  t  g  00200    1  -
688    b  17.92   0.205  u  g  aa   v  0.040  f  f   0  f  g  00280  750  -
689    b  35.00   3.375  u  g   c   h  8.290  f  f   0  t  g  00000    0  -
    \end{verbatim}

    
    \hypertarget{findings}{%
\subparagraph{Findings:}\label{findings}}

We can see that the `?' representing a missing value in row\# 673 has
been replaced by a NaN instead, now enabling numeric operations on the
column.

    \hypertarget{handling-the-missing-values-part-ii}{%
\subsection{4. Handling the missing values (part
ii)}\label{handling-the-missing-values-part-ii}}

We replaced all the question marks with NaNs. This is going to help us
in the next missing value treatment that we are going to perform.

An important question that gets raised here is why are we giving so much
importance to missing values? Can't they be just ignored? Ignoring
missing values can affect the performance of a machine learning model
heavily. While ignoring the missing values our machine learning model
may miss out on information about the dataset that may be useful for its
training. Then, there are many models which cannot handle missing values
implicitly such as LDA.

So, to avoid this problem, we are going to impute the missing values
with a strategy called mean imputation.

    \begin{Verbatim}[commandchars=\\\{\}]
{\color{incolor}In [{\color{incolor}4}]:} \PY{c+c1}{\PYZsh{} Impute the missing values with mean imputation}
        \PY{n}{cc\PYZus{}apps}\PY{o}{.}\PY{n}{fillna}\PY{p}{(}\PY{n}{cc\PYZus{}apps}\PY{o}{.}\PY{n}{mean}\PY{p}{(}\PY{p}{)}\PY{p}{,} \PY{n}{inplace}\PY{o}{=}\PY{k+kc}{True}\PY{p}{)}
        
        \PY{c+c1}{\PYZsh{} Count the number of NaNs in the dataset to verify}
        \PY{c+c1}{\PYZsh{} ... YOUR CODE FOR TASK 4 ...}
        \PY{n}{display}\PY{p}{(}\PY{n}{cc\PYZus{}apps}\PY{o}{.}\PY{n}{isnull}\PY{p}{(}\PY{p}{)}\PY{o}{.}\PY{n}{values}\PY{o}{.}\PY{n}{sum}\PY{p}{(}\PY{p}{)}\PY{p}{)}
\end{Verbatim}

    
    \begin{verbatim}
67
    \end{verbatim}

    
    \hypertarget{findings}{%
\subparagraph{Findings:}\label{findings}}

We can see that there are 67 missing values in our dataset now.

    \hypertarget{handling-the-missing-values-part-iii}{%
\subsection{5. Handling the missing values (part
iii)}\label{handling-the-missing-values-part-iii}}

We have successfully taken care of the missing values present in the
numeric columns. There are still some missing values to be imputed for
columns 0, 1, 3, 4, 5, 6 and 13. All of these columns contain
non-numeric data and this why the mean imputation strategy would not
work here. This needs a different treatment.

We are going to impute these missing values with the most frequent
values as present in the respective columns. This is good practice when
it comes to imputing missing values for categorical data in general.

    \begin{Verbatim}[commandchars=\\\{\}]
{\color{incolor}In [{\color{incolor}5}]:} \PY{c+c1}{\PYZsh{} Iterate over each column of cc\PYZus{}apps}
        \PY{k}{for} \PY{n}{col} \PY{o+ow}{in} \PY{n}{cc\PYZus{}apps}\PY{p}{:}
            \PY{c+c1}{\PYZsh{} Check if the column is of object type}
            \PY{k}{if} \PY{n}{cc\PYZus{}apps}\PY{p}{[}\PY{n}{col}\PY{p}{]}\PY{o}{.}\PY{n}{dtypes} \PY{o}{==} \PY{l+s+s1}{\PYZsq{}}\PY{l+s+s1}{object}\PY{l+s+s1}{\PYZsq{}}\PY{p}{:}
                \PY{c+c1}{\PYZsh{} Impute with the most frequent value}
                \PY{n}{cc\PYZus{}apps} \PY{o}{=} \PY{n}{cc\PYZus{}apps}\PY{o}{.}\PY{n}{fillna}\PY{p}{(}\PY{n}{cc\PYZus{}apps}\PY{p}{[}\PY{n}{col}\PY{p}{]}\PY{o}{.}\PY{n}{value\PYZus{}counts}\PY{p}{(}\PY{p}{)}\PY{o}{.}\PY{n}{index}\PY{p}{[}\PY{l+m+mi}{0}\PY{p}{]}\PY{p}{)}
        
        \PY{c+c1}{\PYZsh{} Count the number of NaNs in the dataset and print the counts to verify}
        \PY{c+c1}{\PYZsh{} ... YOUR CODE FOR TASK 5 ...}
        \PY{n}{display}\PY{p}{(}\PY{n}{cc\PYZus{}apps}\PY{o}{.}\PY{n}{isnull}\PY{p}{(}\PY{p}{)}\PY{o}{.}\PY{n}{values}\PY{o}{.}\PY{n}{sum}\PY{p}{(}\PY{p}{)}\PY{p}{)}
\end{Verbatim}

    
    \begin{verbatim}
0
    \end{verbatim}

    
    \hypertarget{findings}{%
\subparagraph{Findings:}\label{findings}}

We can see that all the missing have been taken care of as there are no
more missing values in our dataset.

    \hypertarget{preprocessing-the-data-part-i}{%
\subsection{6. Preprocessing the data (part
i)}\label{preprocessing-the-data-part-i}}

The missing values are now successfully handled.

There is still some minor but essential data preprocessing needed before
we proceed towards building our machine learning model. We are going to
divide these remaining preprocessing steps into two main tasks:

Convert the non-numeric data into numeric.

Scale the feature values to a uniform range.

First, we will be converting all the non-numeric values into numeric
ones. We do this because not only it results in a faster computation but
also many machine learning models (like XGBoost) (and especially the
ones developed using scikit-learn) require the data to be in a strictly
numeric format. We will do this by using a technique called label
encoding.

    \begin{Verbatim}[commandchars=\\\{\}]
{\color{incolor}In [{\color{incolor}6}]:} \PY{c+c1}{\PYZsh{} Import LabelEncoder}
        \PY{c+c1}{\PYZsh{} ... YOUR CODE FOR TASK 6 ...}
        \PY{k+kn}{from} \PY{n+nn}{sklearn}\PY{n+nn}{.}\PY{n+nn}{preprocessing} \PY{k}{import} \PY{n}{LabelEncoder}
        
        \PY{c+c1}{\PYZsh{} Print DataFrame information before Preprocessing}
        \PY{n}{cc\PYZus{}apps\PYZus{}info} \PY{o}{=} \PY{n}{cc\PYZus{}apps}\PY{o}{.}\PY{n}{info}\PY{p}{(}\PY{p}{)}
        \PY{n}{display}\PY{p}{(}\PY{n}{cc\PYZus{}apps\PYZus{}info}\PY{p}{)}
        
        \PY{c+c1}{\PYZsh{} Instantiate LabelEncoder}
        \PY{c+c1}{\PYZsh{} ... YOUR CODE FOR TASK 6 ...}
        \PY{n}{le} \PY{o}{=} \PY{n}{LabelEncoder}\PY{p}{(}\PY{p}{)}
        
        \PY{c+c1}{\PYZsh{} Iterate over all the values of each column and extract their dtypes}
        \PY{k}{for} \PY{n}{col} \PY{o+ow}{in} \PY{n}{cc\PYZus{}apps}\PY{p}{:}
            \PY{c+c1}{\PYZsh{} Compare if the dtype is object}
            \PY{k}{if} \PY{n}{cc\PYZus{}apps}\PY{p}{[}\PY{n}{col}\PY{p}{]}\PY{o}{.}\PY{n}{dtypes} \PY{o}{==} \PY{l+s+s1}{\PYZsq{}}\PY{l+s+s1}{object}\PY{l+s+s1}{\PYZsq{}}\PY{p}{:}
            \PY{c+c1}{\PYZsh{} Use LabelEncoder to do the numeric transformation}
                \PY{n}{le}\PY{o}{.}\PY{n}{fit}\PY{p}{(}\PY{n}{cc\PYZus{}apps}\PY{p}{[}\PY{n}{col}\PY{p}{]}\PY{p}{)}
                \PY{n}{cc\PYZus{}apps}\PY{p}{[}\PY{n}{col}\PY{p}{]}\PY{o}{=}\PY{n}{le}\PY{o}{.}\PY{n}{transform}\PY{p}{(}\PY{n}{cc\PYZus{}apps}\PY{p}{[}\PY{n}{col}\PY{p}{]}\PY{p}{)}
        
        \PY{c+c1}{\PYZsh{} Print DataFrame information after Preprocessing}
        \PY{n}{cc\PYZus{}apps\PYZus{}info} \PY{o}{=} \PY{n}{cc\PYZus{}apps}\PY{o}{.}\PY{n}{info}\PY{p}{(}\PY{p}{)}
        \PY{n}{display}\PY{p}{(}\PY{n}{cc\PYZus{}apps\PYZus{}info}\PY{p}{)}
\end{Verbatim}

    \begin{Verbatim}[commandchars=\\\{\}]
<class 'pandas.core.frame.DataFrame'>
RangeIndex: 690 entries, 0 to 689
Data columns (total 16 columns):
0     690 non-null object
1     690 non-null object
2     690 non-null float64
3     690 non-null object
4     690 non-null object
5     690 non-null object
6     690 non-null object
7     690 non-null float64
8     690 non-null object
9     690 non-null object
10    690 non-null int64
11    690 non-null object
12    690 non-null object
13    690 non-null object
14    690 non-null int64
15    690 non-null object
dtypes: float64(2), int64(2), object(12)
memory usage: 86.3+ KB

    \end{Verbatim}

    
    \begin{verbatim}
None
    \end{verbatim}

    
    \begin{Verbatim}[commandchars=\\\{\}]
<class 'pandas.core.frame.DataFrame'>
RangeIndex: 690 entries, 0 to 689
Data columns (total 16 columns):
0     690 non-null int64
1     690 non-null int64
2     690 non-null float64
3     690 non-null int64
4     690 non-null int64
5     690 non-null int64
6     690 non-null int64
7     690 non-null float64
8     690 non-null int64
9     690 non-null int64
10    690 non-null int64
11    690 non-null int64
12    690 non-null int64
13    690 non-null int64
14    690 non-null int64
15    690 non-null int64
dtypes: float64(2), int64(14)
memory usage: 86.3 KB

    \end{Verbatim}

    
    \begin{verbatim}
None
    \end{verbatim}

    
    \hypertarget{findings}{%
\subparagraph{Findings:}\label{findings}}

The first output shows that our data had 12 columns of object dtype and
after the process, the second output shows that all these columns have
been converted to int64 dtype.

    \hypertarget{preprocessing-the-data-part-ii}{%
\subsection{7. Preprocessing the data (part
ii)}\label{preprocessing-the-data-part-ii}}

We have successfully converted all the non-numeric values to numeric
ones.

Now, let's try to understand what these scaled values mean in the real
world. Let's use CreditScore as an example. The credit score of a person
is their creditworthiness based on their credit history. The higher this
number, the more financially trustworthy a person is considered to be.
So, a CreditScore of 1 is the highest since we're rescaling all the
values to the range of 0-1.

Also, features like DriversLicense and ZipCode are not as important as
the other features in the dataset for predicting credit card approvals.
We should drop them to design our machine learning model with the best
set of features. This is often called feature engineering or, more
specifically, feature selection.

    \begin{Verbatim}[commandchars=\\\{\}]
{\color{incolor}In [{\color{incolor}7}]:} \PY{c+c1}{\PYZsh{} Import MinMaxScaler}
        \PY{c+c1}{\PYZsh{} ... YOUR CODE FOR TASK 7 ...}
        \PY{k+kn}{from} \PY{n+nn}{sklearn}\PY{n+nn}{.}\PY{n+nn}{preprocessing} \PY{k}{import} \PY{n}{MinMaxScaler}
        
        \PY{c+c1}{\PYZsh{} Print summary statistics before preprocessing}
        \PY{k+kn}{from} \PY{n+nn}{scipy} \PY{k}{import} \PY{n}{stats}
        \PY{n}{cc\PYZus{}apps\PYZus{}description} \PY{o}{=} \PY{n}{stats}\PY{o}{.}\PY{n}{describe}\PY{p}{(}\PY{n}{cc\PYZus{}apps}\PY{p}{)}
        \PY{n}{display}\PY{p}{(}\PY{n}{cc\PYZus{}apps\PYZus{}description}\PY{o}{.}\PY{n}{minmax}\PY{p}{)}
        
        \PY{c+c1}{\PYZsh{} Drop features 11 and 13 and convert the DataFrame to a NumPy array}
        \PY{n}{cc\PYZus{}apps} \PY{o}{=} \PY{n}{cc\PYZus{}apps}\PY{o}{.}\PY{n}{drop}\PY{p}{(}\PY{p}{[}\PY{l+m+mi}{11}\PY{p}{,} \PY{l+m+mi}{13}\PY{p}{]}\PY{p}{,} \PY{n}{axis}\PY{o}{=}\PY{l+m+mi}{1}\PY{p}{)}
        \PY{n}{cc\PYZus{}apps} \PY{o}{=} \PY{n}{cc\PYZus{}apps}\PY{o}{.}\PY{n}{values}
        
        \PY{c+c1}{\PYZsh{} Segregate features and labels into separate variables}
        \PY{n}{X}\PY{p}{,}\PY{n}{y} \PY{o}{=} \PY{n}{cc\PYZus{}apps}\PY{p}{[}\PY{p}{:}\PY{p}{,}\PY{l+m+mi}{0}\PY{p}{:}\PY{o}{\PYZhy{}}\PY{l+m+mi}{1}\PY{p}{]} \PY{p}{,} \PY{n}{cc\PYZus{}apps}\PY{p}{[}\PY{p}{:}\PY{p}{,}\PY{o}{\PYZhy{}}\PY{l+m+mi}{1}\PY{p}{]}
        
        \PY{c+c1}{\PYZsh{} Instantiate MinMaxScaler and use it to rescale}
        \PY{n}{scaler} \PY{o}{=} \PY{n}{MinMaxScaler}\PY{p}{(}\PY{n}{feature\PYZus{}range}\PY{o}{=}\PY{p}{(}\PY{l+m+mi}{0}\PY{p}{,}\PY{l+m+mi}{1}\PY{p}{)}\PY{p}{)}
        \PY{n}{rescaledX} \PY{o}{=} \PY{n}{scaler}\PY{o}{.}\PY{n}{fit\PYZus{}transform}\PY{p}{(}\PY{n}{X}\PY{p}{)}
        
        \PY{c+c1}{\PYZsh{} Print summary statistics after preprocessing}
        \PY{n}{cc\PYZus{}apps\PYZus{}description} \PY{o}{=} \PY{n}{stats}\PY{o}{.}\PY{n}{describe}\PY{p}{(}\PY{n}{rescaledX}\PY{p}{)}
        \PY{n}{display}\PY{p}{(}\PY{n}{cc\PYZus{}apps\PYZus{}description}\PY{o}{.}\PY{n}{minmax}\PY{p}{)}
\end{Verbatim}

    
    \begin{verbatim}
(array([0., 0., 0., 0., 0., 0., 0., 0., 0., 0., 0., 0., 0., 0., 0., 0.]),
 array([1.00e+00, 3.49e+02, 2.80e+01, 3.00e+00, 3.00e+00, 1.40e+01,
        9.00e+00, 2.85e+01, 1.00e+00, 1.00e+00, 6.70e+01, 1.00e+00,
        2.00e+00, 1.70e+02, 1.00e+05, 1.00e+00]))
    \end{verbatim}

    
    
    \begin{verbatim}
(array([0., 0., 0., 0., 0., 0., 0., 0., 0., 0., 0., 0., 0.]),
 array([1., 1., 1., 1., 1., 1., 1., 1., 1., 1., 1., 1., 1.]))
    \end{verbatim}

    
    \hypertarget{findings}{%
\subparagraph{Findings:}\label{findings}}

Each of the outputs shows the min and max values for each column.
Between the first and the second output, we can see that the values have
been scaled between 0 and 1 as the max values are now 1 for all columns.
Moreover, the size of the arrays has reduced, indicating the two dropped
columns

    \hypertarget{splitting-the-dataset-into-train-and-test-sets}{%
\subsection{8. Splitting the dataset into train and test
sets}\label{splitting-the-dataset-into-train-and-test-sets}}

Now that we have our data in a machine learning modeling-friendly shape,
we are really ready to proceed towards creating a machine learning model
to predict which credit card applications will be accepted and which
will be rejected.

First, we will split our data into train set and test set to prepare our
data for two different phases of machine learning modeling: training and
testing.

    \begin{Verbatim}[commandchars=\\\{\}]
{\color{incolor}In [{\color{incolor}8}]:} \PY{c+c1}{\PYZsh{} Import train\PYZus{}test\PYZus{}split}
        \PY{c+c1}{\PYZsh{} ... YOUR CODE FOR TASK 8 ...}
        \PY{k+kn}{from} \PY{n+nn}{sklearn}\PY{n+nn}{.}\PY{n+nn}{model\PYZus{}selection} \PY{k}{import} \PY{n}{train\PYZus{}test\PYZus{}split}
        
        \PY{c+c1}{\PYZsh{} Split into train and test sets}
        \PY{n}{X\PYZus{}train}\PY{p}{,} \PY{n}{X\PYZus{}test}\PY{p}{,} \PY{n}{y\PYZus{}train}\PY{p}{,} \PY{n}{y\PYZus{}test} \PY{o}{=} \PY{n}{train\PYZus{}test\PYZus{}split}\PY{p}{(}\PY{n}{rescaledX}\PY{p}{,}
                                        \PY{n}{y}\PY{p}{,}
                                        \PY{n}{test\PYZus{}size}\PY{o}{=}\PY{l+m+mf}{0.3}\PY{p}{,}
                                        \PY{n}{random\PYZus{}state}\PY{o}{=}\PY{l+m+mi}{0}\PY{p}{)}
\end{Verbatim}

    \hypertarget{findings}{%
\subparagraph{Findings:}\label{findings}}

What we did here was split our data into test and train splits. I have
used 30\% as a trial value because of my past experience working with
the Titanic dataset in my Artificial Intelligence course. No other
findings to be discussed here as such.

    \hypertarget{fitting-a-logistic-regression-model-to-the-train-set}{%
\subsection{9. Fitting a logistic regression model to the train
set}\label{fitting-a-logistic-regression-model-to-the-train-set}}

Essentially, predicting if a credit card application will be approved or
not is a classification task. According to UCI, our dataset contains
more instances that correspond to ``Denied'' status than instances
corresponding to ``Approved'' status. Specifically, out of 690
instances, there are 383 (55.5\%) applications that got denied and 307
(44.5\%) applications that got approved.

This gives us a benchmark. A good machine learning model should be able
to accurately predict the status of the applications with respect to
these statistics.

Which model should we pick? A question to ask is: are the features that
affect the credit card approval decision process correlated with each
other? Although we can measure correlation, that is outside the scope of
this notebook, so we'll rely on our intuition that they indeed are
correlated for now. Because of this correlation, we'll take advantage of
the fact that generalized linear models perform well in these cases.
Let's start our machine learning modeling with a Logistic Regression
model (a generalized linear model).

    \begin{Verbatim}[commandchars=\\\{\}]
{\color{incolor}In [{\color{incolor}9}]:} \PY{c+c1}{\PYZsh{} Import LogisticRegression}
        \PY{c+c1}{\PYZsh{} ... YOUR CODE FOR TASK 9 ...}
        \PY{k+kn}{from} \PY{n+nn}{sklearn}\PY{n+nn}{.}\PY{n+nn}{linear\PYZus{}model} \PY{k}{import} \PY{n}{LogisticRegression}
        
        \PY{c+c1}{\PYZsh{} Instantiate a LogisticRegression classifier with default parameter values}
        \PY{n}{logreg} \PY{o}{=} \PY{n}{LogisticRegression}\PY{p}{(}\PY{n}{solver}\PY{o}{=}\PY{l+s+s1}{\PYZsq{}}\PY{l+s+s1}{lbfgs}\PY{l+s+s1}{\PYZsq{}}\PY{p}{)}
        
        \PY{c+c1}{\PYZsh{} Fit logreg to the train set}
        \PY{c+c1}{\PYZsh{} ... YOUR CODE FOR TASK 9 ...}
        \PY{n}{logreg}\PY{o}{.}\PY{n}{fit}\PY{p}{(}\PY{n}{X\PYZus{}train}\PY{p}{,} \PY{n}{y\PYZus{}train}\PY{p}{)}
\end{Verbatim}

\begin{Verbatim}[commandchars=\\\{\}]
{\color{outcolor}Out[{\color{outcolor}9}]:} LogisticRegression(C=1.0, class\_weight=None, dual=False, fit\_intercept=True,
                  intercept\_scaling=1, max\_iter=100, multi\_class='warn',
                  n\_jobs=None, penalty='l2', random\_state=None, solver='lbfgs',
                  tol=0.0001, verbose=0, warm\_start=False)
\end{Verbatim}
            
    \hypertarget{findings}{%
\subparagraph{Findings:}\label{findings}}

We have got here a model that has been trained on the test attributes
and learned the training labels.

    \hypertarget{making-predictions-and-evaluating-performance}{%
\subsection{10. Making predictions and evaluating
performance}\label{making-predictions-and-evaluating-performance}}

But how well does our model perform?

We will now evaluate our model on the test set with respect to
classification accuracy. But we will also take a look the model's
confusion matrix. In the case of predicting credit card applications, it
is equally important to see if our machine learning model is able to
predict the approval status of the applications as denied that
originally got denied. If our model is not performing well in this
aspect, then it might end up approving the application that should have
been approved. The confusion matrix helps us to view our model's
performance from these aspects.

    \begin{Verbatim}[commandchars=\\\{\}]
{\color{incolor}In [{\color{incolor}10}]:} \PY{c+c1}{\PYZsh{} Import confusion\PYZus{}matrix}
         \PY{c+c1}{\PYZsh{} ... YOUR CODE FOR TASK 10 ...}
         \PY{k+kn}{from} \PY{n+nn}{sklearn}\PY{n+nn}{.}\PY{n+nn}{metrics} \PY{k}{import} \PY{n}{confusion\PYZus{}matrix}
         
         \PY{c+c1}{\PYZsh{} Use logreg to predict instances from the test set and store it}
         \PY{n}{y\PYZus{}pred} \PY{o}{=} \PY{n}{logreg}\PY{o}{.}\PY{n}{predict}\PY{p}{(}\PY{n}{X\PYZus{}test}\PY{p}{)}
         
         \PY{c+c1}{\PYZsh{} Get the accuracy score of logreg model and print it}
         \PY{n}{acc\PYZus{}lr} \PY{o}{=} \PY{n}{logreg}\PY{o}{.}\PY{n}{score}\PY{p}{(}\PY{n}{X\PYZus{}test}\PY{p}{,} \PY{n}{y\PYZus{}test}\PY{p}{)}
         \PY{n+nb}{print}\PY{p}{(}\PY{l+s+s2}{\PYZdq{}}\PY{l+s+s2}{Accuracy of logistic regression classifier: }\PY{l+s+s2}{\PYZdq{}}\PY{p}{,} \PY{n}{acc\PYZus{}lr}\PY{p}{)}
         
         \PY{c+c1}{\PYZsh{} Print the confusion matrix of the logreg model}
         \PY{c+c1}{\PYZsh{} ... YOUR CODE FOR TASK 10 ...}
         \PY{n}{confusion\PYZus{}matrix}\PY{p}{(}\PY{n}{y\PYZus{}test}\PY{p}{,}\PY{n}{y\PYZus{}pred}\PY{p}{)}
\end{Verbatim}

    \begin{Verbatim}[commandchars=\\\{\}]
Accuracy of logistic regression classifier:  0.8454106280193237

    \end{Verbatim}

\begin{Verbatim}[commandchars=\\\{\}]
{\color{outcolor}Out[{\color{outcolor}10}]:} array([[76, 14],
                [18, 99]])
\end{Verbatim}
            
    \hypertarget{findings}{%
\subparagraph{Findings:}\label{findings}}

The accuracy is fairly high, at 84.5\% The confusion matrix shows that
there are 76 True Negatives, 14 False Positives, 18 False Negatives, and
99 True Positives.

    \hypertarget{grid-searching-and-making-the-model-perform-better}{%
\subsection{11. Grid searching and making the model perform
better}\label{grid-searching-and-making-the-model-perform-better}}

Our model was pretty good! It was able to yield an accuracy score of
almost 84\%.

For the confusion matrix, the first element of the of the first row of
the confusion matrix denotes the true negatives meaning the number of
negative instances (denied applications) predicted by the model
correctly. And the last element of the second row of the confusion
matrix denotes the true positives meaning the number of positive
instances (approved applications) predicted by the model correctly.

Let's see if we can do better. We can perform a grid search of the model
parameters to improve the model's ability to predict credit card
approvals.

scikit-learn's implementation of logistic regression consists of
different hyperparameters but we will grid search over the following
two:

tol

max\_iter

    \begin{Verbatim}[commandchars=\\\{\}]
{\color{incolor}In [{\color{incolor}11}]:} \PY{c+c1}{\PYZsh{} Import GridSearchCV}
         \PY{c+c1}{\PYZsh{} ... YOUR CODE FOR TASK 11 ...}
         \PY{k+kn}{from} \PY{n+nn}{sklearn}\PY{n+nn}{.}\PY{n+nn}{model\PYZus{}selection} \PY{k}{import} \PY{n}{GridSearchCV}
         
         \PY{c+c1}{\PYZsh{} Define the grid of values for tol and max\PYZus{}iter}
         \PY{n}{tol} \PY{o}{=} \PY{p}{[}\PY{l+m+mf}{0.01}\PY{p}{,}\PY{l+m+mf}{0.001}\PY{p}{,}\PY{l+m+mf}{0.0001}\PY{p}{]}
         \PY{n}{max\PYZus{}iter} \PY{o}{=} \PY{p}{[}\PY{l+m+mi}{100}\PY{p}{,}\PY{l+m+mi}{150}\PY{p}{,}\PY{l+m+mi}{200}\PY{p}{]}
         
         \PY{c+c1}{\PYZsh{} Create a dictionary where tol and max\PYZus{}iter are keys and the lists of their values are corresponding values}
         \PY{n}{param\PYZus{}grid} \PY{o}{=} \PY{n+nb}{dict}\PY{p}{(}\PY{n}{tol}\PY{o}{=}\PY{n}{tol}\PY{p}{,}\PY{n}{max\PYZus{}iter}\PY{o}{=}\PY{n}{max\PYZus{}iter}\PY{p}{)}
\end{Verbatim}

    \hypertarget{findings}{%
\subparagraph{Findings:}\label{findings}}

Param grid here is a dictionary that contains our search grid with
values for tol ranging from 0.0001 to 0.001, and for max\_iter ranging
from 50 to 250.

    \hypertarget{finding-the-best-performing-model}{%
\subsection{12. Finding the best performing
model}\label{finding-the-best-performing-model}}

We have defined the grid of hyperparameter values and converted them
into a single dictionary format which GridSearchCV() expects as one of
its parameters. Now, we will begin the grid search to see which values
perform best.

We will instantiate GridSearchCV() with our earlier logreg model with
all the data we have. Instead of passing train and test sets, we will
supply rescaledX and y. We will also instruct GridSearchCV() to perform
a cross-validation of five folds.

We'll end the notebook by storing the best-achieved score and the
respective best parameters.

While building this credit card predictor, we tackled some of the most
widely-known preprocessing steps such as scaling, label encoding, and
missing value imputation. We finished with some machine learning to
predict if a person's application for a credit card would get approved
or not given some information about that person.

    \begin{Verbatim}[commandchars=\\\{\}]
{\color{incolor}In [{\color{incolor}12}]:} \PY{c+c1}{\PYZsh{} Instantiate GridSearchCV with the required parameters}
         \PY{n}{grid\PYZus{}model} \PY{o}{=} \PY{n}{GridSearchCV}\PY{p}{(}\PY{n}{estimator}\PY{o}{=}\PY{n}{logreg}\PY{p}{,} \PY{n}{param\PYZus{}grid}\PY{o}{=}\PY{n}{param\PYZus{}grid}\PY{p}{,} \PY{n}{cv}\PY{o}{=}\PY{l+m+mi}{5}\PY{p}{)}
         
         \PY{c+c1}{\PYZsh{} Fit data to grid\PYZus{}model}
         \PY{n}{grid\PYZus{}model\PYZus{}result} \PY{o}{=} \PY{n}{grid\PYZus{}model}\PY{o}{.}\PY{n}{fit}\PY{p}{(}\PY{n}{rescaledX}\PY{p}{,} \PY{n}{y}\PY{p}{)}
         
         \PY{c+c1}{\PYZsh{} Summarize results}
         \PY{n}{best\PYZus{}score}\PY{p}{,} \PY{n}{best\PYZus{}params} \PY{o}{=} \PY{n}{grid\PYZus{}model\PYZus{}result}\PY{o}{.}\PY{n}{best\PYZus{}score\PYZus{}} \PY{p}{,} \PY{n}{grid\PYZus{}model\PYZus{}result}\PY{o}{.}\PY{n}{best\PYZus{}params\PYZus{}}
         \PY{n+nb}{print}\PY{p}{(}\PY{l+s+s2}{\PYZdq{}}\PY{l+s+s2}{Best: }\PY{l+s+si}{\PYZpc{}f}\PY{l+s+s2}{ using }\PY{l+s+si}{\PYZpc{}s}\PY{l+s+s2}{\PYZdq{}} \PY{o}{\PYZpc{}} \PY{p}{(}\PY{n}{best\PYZus{}score}\PY{p}{,} \PY{n}{best\PYZus{}params}\PY{p}{)}\PY{p}{)}
\end{Verbatim}

    \begin{Verbatim}[commandchars=\\\{\}]
Best: 0.850725 using \{'max\_iter': 100, 'tol': 0.01\}

    \end{Verbatim}

    \hypertarget{findings}{%
\subparagraph{Findings:}\label{findings}}

It can be observed that out of all the possible combinations of tol with
max\_iter, the best outcome is achieved with tol=0.01 and max\_iter=100
with highest accuracy of 85.07\%

    \hypertarget{fitting-another-new-model-of-your-choice-to-the-same-train-set}{%
\subsection{13. Fitting another new model of your choice to the same
train
set}\label{fitting-another-new-model-of-your-choice-to-the-same-train-set}}

Same description as Question 9

    \begin{Verbatim}[commandchars=\\\{\}]
{\color{incolor}In [{\color{incolor}13}]:} \PY{c+c1}{\PYZsh{} Import RidgeClassifier}
         \PY{k+kn}{from} \PY{n+nn}{sklearn}\PY{n+nn}{.}\PY{n+nn}{linear\PYZus{}model} \PY{k}{import} \PY{n}{RidgeClassifier}
         
         \PY{c+c1}{\PYZsh{} Instantiate a RidgeClassifier with default parameter values}
         \PY{n}{ridge\PYZus{}reg} \PY{o}{=} \PY{n}{RidgeClassifier}\PY{p}{(}\PY{p}{)}
         
         \PY{c+c1}{\PYZsh{} Fit ridge\PYZus{}reg to the train set}
         \PY{n}{ridge\PYZus{}reg}\PY{o}{.}\PY{n}{fit}\PY{p}{(}\PY{n}{X\PYZus{}train}\PY{p}{,} \PY{n}{y\PYZus{}train}\PY{p}{)}
\end{Verbatim}

\begin{Verbatim}[commandchars=\\\{\}]
{\color{outcolor}Out[{\color{outcolor}13}]:} RidgeClassifier(alpha=1.0, class\_weight=None, copy\_X=True, fit\_intercept=True,
                 max\_iter=None, normalize=False, random\_state=None, solver='auto',
                 tol=0.001)
\end{Verbatim}
            
    \hypertarget{findings}{%
\subparagraph{Findings:}\label{findings}}

We have got here a model that has been trained on the test attributes
and learned the training labels.

    \hypertarget{making-predictions-and-evaluating-performance-for-new-model}{%
\subsection{14. Making predictions and evaluating performance for new
model}\label{making-predictions-and-evaluating-performance-for-new-model}}

Same description as Question 10

    \begin{Verbatim}[commandchars=\\\{\}]
{\color{incolor}In [{\color{incolor}14}]:} \PY{c+c1}{\PYZsh{} Use ridge\PYZus{}reg to predict instances from the test set and store it}
         \PY{n}{y\PYZus{}pred\PYZus{}r} \PY{o}{=} \PY{n}{ridge\PYZus{}reg}\PY{o}{.}\PY{n}{predict}\PY{p}{(}\PY{n}{X\PYZus{}test}\PY{p}{)}
         
         \PY{c+c1}{\PYZsh{} Get the accuracy score of ridge\PYZus{}reg model and print it}
         \PY{n}{acc\PYZus{}rr} \PY{o}{=} \PY{n}{ridge\PYZus{}reg}\PY{o}{.}\PY{n}{score}\PY{p}{(}\PY{n}{X\PYZus{}test}\PY{p}{,} \PY{n}{y\PYZus{}test}\PY{p}{)}
         \PY{n+nb}{print}\PY{p}{(}\PY{l+s+s2}{\PYZdq{}}\PY{l+s+s2}{Accuracy of RidgeClassifier: }\PY{l+s+s2}{\PYZdq{}}\PY{p}{,} \PY{n}{acc\PYZus{}rr}\PY{p}{)}
         
         \PY{c+c1}{\PYZsh{} Print the confusion matrix of the ridge\PYZus{}reg model}
         \PY{n}{confusion\PYZus{}matrix}\PY{p}{(}\PY{n}{y\PYZus{}test}\PY{p}{,}\PY{n}{y\PYZus{}pred\PYZus{}r}\PY{p}{)}
\end{Verbatim}

    \begin{Verbatim}[commandchars=\\\{\}]
Accuracy of RidgeClassifier:  0.855072463768116

    \end{Verbatim}

\begin{Verbatim}[commandchars=\\\{\}]
{\color{outcolor}Out[{\color{outcolor}14}]:} array([[84,  6],
                [24, 93]])
\end{Verbatim}
            
    \hypertarget{findings}{%
\subparagraph{Findings:}\label{findings}}

The accuracy is fairly high, at 85.5\% The confusion matrix shows that
there are 84 True Negatives, 6 False Positives, 24 False Negatives, and
93 True Positives.

    \hypertarget{which-model-is-better-in-terms-of-classification-accuracy}{%
\subsection{15. Which model is better in terms of Classification
Accuracy?}\label{which-model-is-better-in-terms-of-classification-accuracy}}

Compare two models and explain which one is better and why?

    \begin{Verbatim}[commandchars=\\\{\}]
{\color{incolor}In [{\color{incolor}15}]:} \PY{c+c1}{\PYZsh{} printing accuracies}
         \PY{n+nb}{print}\PY{p}{(}\PY{l+s+s2}{\PYZdq{}}\PY{l+s+s2}{Logistic Regression accuracy is}\PY{l+s+s2}{\PYZdq{}}\PY{p}{,} \PY{n}{acc\PYZus{}lr}\PY{p}{)}
         \PY{n+nb}{print}\PY{p}{(}\PY{l+s+s2}{\PYZdq{}}\PY{l+s+s2}{Ridge Classifier accuracy is}\PY{l+s+s2}{\PYZdq{}}\PY{p}{,} \PY{n}{acc\PYZus{}rr}\PY{p}{)}
         
         \PY{c+c1}{\PYZsh{} comparing accuracy}
         \PY{k}{if} \PY{n}{acc\PYZus{}lr} \PY{o}{\PYZgt{}} \PY{n}{acc\PYZus{}rr}\PY{p}{:}
             \PY{n+nb}{print}\PY{p}{(}\PY{l+s+s2}{\PYZdq{}}\PY{l+s+s2}{Logistic Regression is better!}\PY{l+s+s2}{\PYZdq{}}\PY{p}{)}
         \PY{k}{else}\PY{p}{:}
             \PY{n+nb}{print}\PY{p}{(}\PY{l+s+s2}{\PYZdq{}}\PY{l+s+s2}{Ridge Classifier is better!}\PY{l+s+s2}{\PYZdq{}}\PY{p}{)}
\end{Verbatim}

    \begin{Verbatim}[commandchars=\\\{\}]
Logistic Regression accuracy is 0.8454106280193237
Ridge Classifier accuracy is 0.855072463768116
Ridge Classifier is better

    \end{Verbatim}

    \hypertarget{findings}{%
\subparagraph{Findings:}\label{findings}}

As we can see, Ridge Classifier is better because of it's higher
Accuracy, as compared to Logistic Regression.


    % Add a bibliography block to the postdoc
    
    
    
    \end{document}
